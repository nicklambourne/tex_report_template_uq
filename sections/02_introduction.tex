\section{Introduction} % Nick
% The introduction should describe the goal of the project (high level), the motivation, and the relevant context.

In 2011 technology venture capitalist Marc Andreessen rightly observed that ``software is eating the world" \cite{andreessen_2011}. However, the means by which this software is provided has shifted dramatically with the increasing availability of internet connectivity, with native applications giving way to the rise of cloud computing \cite{goasduff_2021}. This massive shift has led to the increasing prominence of the technologies known by the umbrella term ``load-balancing". This technology has taken the form of hardware and (optionally) software that sits in front of (horizontally distributed) application servers and is responsible for directing requests to each of them in an attempt to create ``balance" between each server's workload, minimising latency and maximising throughput. With the required underlying software and hardware in place, more sophisticated load processing can be achieved by the implementation of a variety of load-balancing algorithms responsible for deciding where each request is routed \cite{loadBalancingAlgorithms}. \\

Parallel to the development of load-balancing technology, the underlying technology powering the networks they run on has also been undergoing something of a revolution, again with software at the forefront. The rise of Software-Defined Networks (SDN) has shifted the responsibility of controlling networks from technicians physically changing network hardware to highly configurable software operating on top of stable, generic hardware \cite{sdn_wp}. \\

These two development approaches have yet to substantially impact one another, with the vast majority of load balancing still happening on dedicated hardware (e.g. BIG-IP \cite{BIG-IP}), or on generic hardware but with dedicated software (e.g. Amazon's ELB \cite{awsALB}) \cite{market_share}. This proposal suggests the exploration of the limits of combining these two complementary technologies by implementing load balancing at the network level using an SDN controller (the Open Network Operating System; ONOS \cite{ONOS}). In theory, this should allow for more efficient resource usage by doing away with the need for any dedicated load-balancing hardware, dedicated or generic, (excepting the SDN controller and the network switch which are assumed to have already been part of the system). \\

Appropriate background information required for understanding the relevant concepts is covered in Section \ref{sec:bg}, including the technologies required for implementing such a system, as well as the infrastructure required to test it, and an exploration of various load-balancing algorithms. Section \ref{sec:related_work} explores in greater depth that existing work most relevant to our proposed solution including several proof-of-concept implementations provided as examples in the ONOS monorepository \cite{onos_repo}. The specific goal and the constituent milestones (as far as they are currently understood) expected to be required to implement it are detailed in Section \ref{sec:aim}. These milestones have also been allocated to group members and given an estimated completion time, graphed visually in Section \ref{sec:plan}. More detailed implementation details related to the milestones are explored in Section \ref{sec:method} and work already completed towards achieving them has been covered in Section \ref{sec:progress}.


